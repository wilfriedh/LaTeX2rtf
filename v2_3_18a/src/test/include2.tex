\subsection{This is the first subsection in file \texttt{include2.tex}}
\label{section21}

Here is some text to create a paragraph or two so that we
can see if this works or not.  It will be interesting to see
if the labels work properly.  As I create these testing files
I realize that I need to add parsing support for \verb#\input#
and \verb#\include# in the getSection function.   Furthermore,
after reading the \LaTeX{} book, I see that \verb#\include# 
files will all start on a new page.  This is not the case for
\verb#\input# files.  Here is a reference to the next subsection \ref{section22}.

\subsection{This is the second subsection in file \texttt{include2.tex}}
\label{section22}

Here is some text to create a paragraph or two so that we
can see if this works or not.  It will be interesting to see
if the labels work properly.  As I create these testing files
I realize that I need to add parsing support for \verb#\input#
and \verb#\include# in the getSection function.   Furthermore,
after reading the \LaTeX{} book, I see that \verb#\include# 
files will all start on a new page.  This is not the case for
\verb#\input# files.  Here is a reference to the previous subsection \ref{section21}.
Here is a reference to the first section in \texttt{include1.tex}
\ref{section11}.

Now we include another file from within an included file

\section{Testing how including files within other files}
\label{section3}

The trick here is to test the various ways that files might be included
within another file.  

\subsubsection{This is the first subsubsection from file \texttt{include3.tex}}
\label{section31}

Here is some text to create a paragraph or two so that we
can see if this works or not.  It will be interesting to see
if the labels work properly.  As I create these testing files
I realize that I need to add parsing support for \verb#\input#
and \verb#\include# in the getSection function.   Furthermore,
after reading the \LaTeX{} book, I see that \verb#\include# 
files will all start on a new page.  This is not the case for
\verb#\input# files.  Here is a reference to the next subsubsection \ref{section32}.

\subsubsection{This is the second subsubsection from file \texttt{include3.tex}}
\label{section32}

Here is some text to create a paragraph or two so that we
can see if this works or not.  It will be interesting to see
if the labels work properly.  As I create these testing files
I realize that I need to add parsing support for \verb#\input#
and \verb#\include# in the getSection function.   Furthermore,
after reading the \LaTeX{} book, I see that \verb#\include# 
files will all start on a new page.  This is not the case for
\verb#\input# files.  Here is a reference to the previous subsubsection \ref{section31}.
Here is a reference to the first subsection from \texttt{include1.tex}
\ref{section11}.

This is the content following \verb#\subsubsection{This is the first subsubsection from file \texttt{include3.tex}}
\label{section31}

Here is some text to create a paragraph or two so that we
can see if this works or not.  It will be interesting to see
if the labels work properly.  As I create these testing files
I realize that I need to add parsing support for \verb#\input#
and \verb#\include# in the getSection function.   Furthermore,
after reading the \LaTeX{} book, I see that \verb#\include# 
files will all start on a new page.  This is not the case for
\verb#\input# files.  Here is a reference to the next subsubsection \ref{section32}.

\subsubsection{This is the second subsubsection from file \texttt{include3.tex}}
\label{section32}

Here is some text to create a paragraph or two so that we
can see if this works or not.  It will be interesting to see
if the labels work properly.  As I create these testing files
I realize that I need to add parsing support for \verb#\input#
and \verb#\include# in the getSection function.   Furthermore,
after reading the \LaTeX{} book, I see that \verb#\include# 
files will all start on a new page.  This is not the case for
\verb#\input# files.  Here is a reference to the previous subsubsection \ref{section31}.
Here is a reference to the first subsection from \texttt{include1.tex}
\ref{section11}.
#.  There should be no
pagebreak.

\input include4
This is the content following \verb#\input include4 #.  There should be no
pagebreak.

